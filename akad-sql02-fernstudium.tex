%
% Erstellt von Daniel Falkner
% daniel.falkner@akad.de
% 
\documentclass[xcolor=dvipsnames]{beamer}
\usepackage[T1]{fontenc}
\usepackage[utf8]{inputenc}
\usepackage[justification=centering,figurename=Abb.]{caption}

\usetheme{Warsaw}
\usecolortheme[named=YellowGreen]{structure}

\newcommand*{\Title}{Das Fernstudium mit \LaTeX{} } %Titel
\subtitle{Wissenschaftliche Arbeiten ohne Haarausfall} %Untertitel
\newcommand*{\Author}{Daniel Falkner} %Name
\institute{AKAD Pinneberg} %Uni
\titlegraphic{\includegraphics[scale=0.2]{akad_logo.png}} %Logo

\title{\Title}
\author{\Author}
\date{\today}

%Pdf Metainformationen
\subject{\Title}
\keywords{}

\begin{document}

%Titelseite
\begin{frame}
    \titlepage
\end{frame}

%Logo auf allen weiteren Folien
%\logo{\includegraphics[scale=0.1]{akad_logo.png}}

%Inhaltsverzeichniss
\frame{\tableofcontents} 

%Ab hier die Folien 
 
% Struktur der Präsentation nach dem dialektische Fünfsatz 
% 1. Thema
% 2. Gegenargumente
% 3. Argument die für Latex sprechen
% 4. Syntese (Urteil, eigener Standpunkt)
% 5. Apell

\section{Was ist \LaTeX{}?}
\begin{frame}
  \frametitle{\LaTeX{} ist ... }
	\begin{itemize}
  		\item kein Naturkautschuk \pause
	  	\item ein Softwarepaket, das die Benutzung des Textsatzprogramms \TeX{} mit Hilfe von Makros vereinfacht \pause
	  	\item Der Name ist eine Abkürzung für \textbf{La}mport \footnote<3>{Entwickler: Leslie Lamport} \textbf{TeX}
	\end{itemize}
\end{frame}

\section{Vorlagen}
\begin{frame}
  \frametitle{Vorlagen}
  \framesubtitle{Erstellt nach AKAD Vorgaben}
	 \begin{block}{Einfach ausprobieren!}
	  \begin{itemize}
  		\item Assignment \\ \url{https://github.com/derdanu/akad-vorlage}
	  	\item Präsentation \\ \url{https://github.com/derdanu/akad-beamer-vorlage}
	  \end{itemize}
  \end{block}	


\end{frame}

\subsection*{Ende}
\begin{frame}
	\begin{block}{}	
		\begin{center}
			Vielen Dank für Ihre Aufmerksamkeit. \\
			\Author{}
		\end{center}	
	\end{block}
\end{frame}

\end{document}


