%
% Erstellt von Daniel Falkner
% daniel.falkner@akad.de
% 
\documentclass[xcolor=dvipsnames]{beamer}
%\documentclass[12pt,handout]{beamer}
\usepackage[T1]{fontenc}
\usepackage[utf8]{inputenc}
\usepackage[ngerman]{isodate}
\usepackage[justification=centering,figurename=Abb.]{caption}
\usepackage{listings}
\lstset{language=[LaTeX]TeX}

\usetheme{Warsaw}
\usecolortheme[named=OliveGreen]{structure}

\newcommand*{\Title}{Das Fernstudium mit \LaTeX{} } %Titel
\subtitle{Wissenschaftliche Arbeiten ohne Haarausfall} %Untertitel
\newcommand*{\Author}{Daniel Falkner} %Name
\institute{AKAD Pinneberg} %Uni
\titlegraphic{\includegraphics[scale=0.2]{akad_logo.png}} %Logo

\title{\Title}
\author{\Author}
\date{\today}

%Pdf Metainformationen
\subject{\Title}
\keywords{}

\begin{document}

%Titelseite
\begin{frame}
    \titlepage
\end{frame}

%Logo auf allen weiteren Folien
%\logo{\includegraphics[scale=0.1]{akad_logo.png}}

%Inhaltsverzeichniss
\frame{\tableofcontents[hideallsubsections]} 

%Ab hier die Folien 
 
% Struktur der Präsentation nach dem dialektische Fünfsatz 
% 1. Thema
% 2. Gegenargumente
% 3. Argument die für Latex sprechen
% 4. Syntese (Urteil, eigener Standpunkt)
% 5. Apell

\section{Was ist \LaTeX{}?}
\begin{frame}
  \frametitle{Was ist \LaTeX{}?}
	\begin{block}{\LaTeX{} ist ... }	
		\begin{itemize}
  			\item kein Naturkautschuk
	  		\item ein Softwarepaket 
	  		\begin{itemize}
	  			\item aufbauend auf \TeX \footnote{Textsatzsystem von Donald E. Knuth} 
	  			\item entwickelt Anfang 1980 von Leslie Lamport
	  		\end{itemize}
		\end{itemize}
	\end{block}
\end{frame}

\section{Nachteile}
\begin{frame}
  \frametitle{Nachteile}
	\begin{alertblock}{}	
		\begin{itemize}
  			\item kein WYSIWYG (What You See Is What You Get)
		\end{itemize}
	\end{alertblock}
\end{frame}

\section{Vorteile}
\begin{frame}
  \frametitle{Vorteile}
 	 \tableofcontents[currentsection]
\end{frame}

\subsection{Einheitlich}
\begin{frame}
  \frametitle{Einheitlich}
	\begin{block}{}	
		\begin{itemize}
  			\item Die Struktur der Dokumente ist einheitlich.
		\end{itemize}
	\end{block}	
\end{frame}

\subsection{Verzeichnisse}
\begin{frame}
  \frametitle{Verzeichnisse}
	\begin{block}{}	
		\begin{itemize}
  			\item Inhaltsverzeichnis
  			\item Abbildungsverzeichnis
  			\item Tabellenverzeichnis
	  		\item Abkürzungsverzeichnis
  			\item Literaturverzeichnis
		\end{itemize}
	\end{block}			
\end{frame}

\subsection{Mathematische Formeln}
\begin{frame}
  \frametitle{Mathematische Formeln}
	\begin{block}{}	
		\begin{itemize}
  			\item $\sqrt[n-1] {\frac{\sqrt{x^2+y^2+2xy}}{\sqrt{x^2}}} = z$ \pause
	  		\item Beispiel: \lstinline!\\frac\{1\}\{2\}! $\Rightarrow$ $\frac{1}{2}$ 
		\end{itemize}
	\end{block}			
\end{frame}

\subsection{Textdateien}
\begin{frame}
  \frametitle{Textdateien}
	\begin{block}{}	
		\begin{itemize}
  			\item \LaTeX{} Dateien sein einfache Textdateien
  			\item keine Editorbindung und Platformübergreifend
	  		\item Anbindung an Versionsverwaltungen (SVN, GIT)
		\end{itemize}
	\end{block}	
\end{frame}

\section{Vorlagen}
\begin{frame}
  \frametitle{Vorlagen}
  \framesubtitle{Erstellt nach AKAD Vorgaben}
	 \begin{block}{Einfach ausprobieren!}
	  \begin{itemize}
  		\item Assignment \\ \url{https://github.com/derdanu/akad-vorlage}
	  	\item Präsentation \\ \url{https://github.com/derdanu/akad-beamer-vorlage}
	  \end{itemize}
  \end{block}	
\end{frame}

\subsection*{Ende}
\begin{frame}
	\begin{block}{}	
		\begin{center}
			Vielen Dank für Ihre Aufmerksamkeit. \\
			\Author{}
		\end{center}	
	\end{block}
\end{frame}

\end{document}


